\section*{Introdução}
\label{cap:introducao}

Atualmente o uso de redes de computadores tem aumentado exponencialmente. Muitas dessas redes são distribuídas de forma geograficamente separadas precisando de uma complexa infra-estrutura de software e hardware para gerenciá-las e conectá-las. Dentre as diversas soluções existem a grade computacional (\emph{grid computing}) possui característica viabiliza essa conexão.

Segundo Dantas \cite{Mangan2006}, pode-se dizer, também, que representa uma forma estendida dos serviços web permitindo que recursos computacionais possam ser compartilhados.

Defini-se grades como uma plataforma computacional heterogênia distribuída geograficamente fornecendo serviços e recursos às organizações participantes da plataforma.

\cite{Mangan2006} O Global Grid Forum (GGF) uma comunidade fórum com milhares de indivíduos representando mais de 400 organizações em mais de 50 países criou e documentou especificações técnicas e experiências de usuários. O GGF definiu grades computacionais como um ambiente persistente o qual abilita aplicações para integrar intrumentos, disponibilizar informações em locações difusas. Desde lá esta não é a única e precisa definição para o conceito de grades. \cite{Kesselman2001} Define um sistema em grade propondo um \emph{checklist} de três pontos.

\begin{enumerate}
	\item coordenar recursos quais não são direcionados para um controle central.
	\item usar protocolos e interfaces padronizados, abertos para propósitos gerais.
	\item oferecer QoS (qualidade de serviço) não triviais tais como: autenticação, escalonamnto de tarefas, disponibilidade.
\end{enumerate}

Uma definição formal do que um sistema em grade pode prover foi definido em \cite{Mangan2006}. Focando na sua semântica, mostrando que grades não são apenas uma modificação de um sistema distribuído convencional. Podem apresentar recursos heterogênicos como sensores e detectores e não apenas nós computacionais. Abaixo uma lista de aspectos que evidenciam uma grade computacional \cite{Cirne2002}

\begin{itemize}
	\item heterogeneidade
	\item alta dispersão geográfica
	\item compartilhamento ( não pode ser dedicado a uma única aplicação )
	\item múltiplos domínios administrativos ( recursos de várias instituições )
	\item controle distribuído 
\end{itemize}

A grade deve estar preparada para lidar com todo o dinamismo e variabilidade, procurando obter a melhor performance possível adaptando-se ao cenário no momento.

Devido à grande escala, ampla distribuição e existência de múltiplos domínios administrativos, a construção de um escalonador de recursos para grades é praticamente inviável, até porque, convencer os administradores dos recursos que compõem a grade abrirem mão do controle dos seus recursos não é uma tarefa nada fácil. Escalonadores têm como características receber solicitações de vários usuários, arbitrando, portanto, entre os usuários, o uso dos recursos controlados.  

\cite{Thomas1996} considera escalonar como um problema de gerenciamento de recursos. Basicamente um mecanismo ou uma política usada para, eficientemente e efetivamente, gerenciar o acesso e uso de um determinado recurso. Porém, de arcordo com o GGF's \cite{M.2002}, escalonamento é o processo de ordenar tarefas sobre os recursos computacionais e ordenar a comunicação entre as tarefas, assim sendo, ambas aplicações e sistemas devem ser escalonadas.
