\section{Metodologia}
\label{cap:metodologia}

Para verificar a viabilidade da exportação de tarefas do AppMan para RMS diferentes, um estudo da especificação DRMAA será realizado visando o desenvolvimento desta integração. Também será realizado um estudo aprofundado no modelo GRAND onde o protótipo foi desenvolvido e um minucioso estudo da implementação atual do AppMan. Além do estudo da DRMAA, serão avaliadas outras formas de integração entre RMS.

Baseado no que já foi estudado sobre AppMan, notou-se que o desenvolvimento proposto na integração será feito na linguagem Java que foi a linguagem de desenvolvimento do AppMan que visa a portabilidade. 

Será feita uma comparação através de métricas para avaliar as vantagens e desvantagens com o próprio escalonador do AppMan. 

O método de pesquisa utilizado está entre um experimento baseando-se no fato de que o modelo GRAND espera a integração com qualquer RMS e um estudo de caso em função da análise da implementação proposta.

Até o presente momento está sendo feito um estudo sobre a DRMAA e onde já foi empregada verificando suas fases bem como o resultado dessas implementações \cite{Templeton}, \cite{Llorente2005} e \cite{Haas2004}.

Será feito uma análise no diagrama de classes do AppMan visando a melhor forma de implementação da solução. Também será feita a identificação das deficências do AppMan com relação ao modelo GRAND.
