\chapter{Aquisição e Instalação do Protótipo AppMan}
\label{anexo:appman}

Tutorial para instalação e utilização do AppMan.

\begin{scriptsize}
\begin{verbatim}
	Para adquirir os fontes do AppMan é necessário efetuar o checkout no servidor onde encontram-se os fontes.

	$ mkdir ~/appman
	$ cd ~/appman	
	$ svn co https://saloon.inf.ufrgs.br/svn/appman/branches/rbrosinha-pp-2006-2/ .
	
	Os fontes devem estar em todos os nós que deseja-se executadar tarefas.
	
	Sobrescreva o arquivo de configuração para o middleware EXEHDA (exehda-services.xml) com o arquivo do apêndice A.
	
	Criar o diretório que guardará as informações de execução.
	
	$ mkdir ~/appman/exehda/log
	
	Inicializar o EXEHDA a partir do diretório log.
	
	Inicialização na máquina base
	
	$ cd ~/appman/exehda/log
	$ ../bin/exehda --profile base
	
	Inicialização nos demais nós
	$ cd ~/appman/exehda/log
	$ ../bin/exehda --profile nodo1
	
	Só após o serviço estar sendo executado em todos os nós, executa-se o AppMan com o arquivo .dag
	no diretório raiz do protótipo.
	
	$ cd ~/appman
	$ exehda/bin/isam-run appman-console -- /home/tonismar/Publico/appman/exemplos/arquivo.dag
	
\end{verbatim}
\end{scriptsize}