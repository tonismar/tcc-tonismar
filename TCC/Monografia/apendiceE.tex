\chapter{Instalação do Servidor LDAP}
\label{anexo:ldap}

Para a instalação do servidor LDAP em um Linux Ubuntu 7.10 os seguintes passos foram executados.

\begin{scriptsize}
\begin{verbatim}
	$ sudo apt-get install slapd ldap-utils db4.2-util
	$ slappasswd (informar a nova senha solicitada)
	a saída vai ser algo parecido com isto {SSHA}d2BamRTgBuhC6SxC0vFGWol31ki8iq5m
	
	edição do arquivo slapd.conf
	$ sudo vi /etc/ldap/slapd.conf
	
	adicionar a linha
	include /home/tonismar/Publico/appman/exehda/extra/exehda.schema
	
	alterar as seguintes linhas
	suffix "dc=exehda"
	rootdn "cn=admin,dc=exehda"
	rootpw senha criada com o comando ladppasswd
	
	edição do arquivo ldap.conf
	$ sudo vi /etc/ldap/ldap.conf
	
	adicionar a linha
	BASE dc=exehda
	
	criação do arquivo init.ldif
	$ sudo vi init.ldif
	
	-------inicio do conteúdo do arquivo------
	dn: dc=exehda
	objectClass: dcObject
	objectClass: organizationalUnit
	dc: exehda
	ou: Configuracao para EXEHDA
	
	dn: cn=admin,dc=exehda
	objectClass: simpleSecurityObject
	objectClass: organizationalRole
	cn: admin
	description: LDAP administrator
	userPassword: {SSHA}d2BamRTgBuhC6SxC0vFGWol31ki8iq5m
	-------fim do conteúdo do arquivo--------
	
	para adicionar as entidades no arquivo acima, pare o daemon LDAP
	$ sudo /etc/init.d/slapd stop
	
	remova o conteúdo que foi automaticamente adicionado na instalação
	$ sudo rm -rf /var/lib/ldap/*
	
	adiciona o novo conteúdo
	$ sudo slapadd -l init.ldif
	
	corrija as permissões no banco de dados
	$ sudo chown -R openldap:openldap /var/lib/ldap
	
	reinicie o daemon LDAP
	$ sudo /etc/init.d/slapd start
	
	para verificar se o conteúdo foi adicionado corretamente
	$ ldapsearch -xLLL -b "dc=exehda"
	
	a saída será algo aproximado ao que segue abaixo
	
	dn: dc=exehda
	objectClass: dcObject
	objectClass: organizationalUnit
	dc: exehda
	ou: Configuracao para EXEHDA

	dn: cn=admin,dc=exehda
	objectClass: simpleSecurityObject
	objectClass: organizationalRole
	cn: admin
	description: LDAP administrator

\end{verbatim}


\end{scriptsize}