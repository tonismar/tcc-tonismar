
%
% inicio do documento
%

% capa
\maketoppage

% folha de rosto
%\maketitle

\thetitlepage
\makeappage

% dedicatoria
%\clearpage
%\begin{flushright}
%\mbox{}\vfill
%{\sffamily\itshape
%``If I have seen farther than others,\\
%it is because I stood on the shoulders of giants.''\\}
%--- \textsc{Sir~Isaac Newton}
%\end{flushright}

% agradecimentos
\chapter*{Agradecimentos}

\begin{itemize}
	\item Nada se consegue sozinho, a família é a nossa base para tudo. Obrigado pai, mãe, tios, tias, irmã e todos os outros que, por algum motivo, vieram juntar-se a minha família.
	\item Aos amigos de faculdade em especial para Cristiane Ávila, Taís Américo Almeida e Wagner Nobres que, direta ou indiretamente, contribuíram para conclusão deste trabalho.
	\item Aos demais amigos que apoiaram e confiaram no meu potencial me estimulando e ajudando.
	\item Aos professores que me acompanharam e auxiliaram nesta longa caminhada, em especial a professora Patrícia Kayser, minha orientadora, que me ajudou em toda confecção deste trabalho.
	\item Aos membros do grupo do projeto GRAND que tive contato em busca de auxílio nas dúvidas encontradas.
	\item A Márcia Barbosa Abreu pela paciência, compreensão e apoio nos momentos complicados.
	\item E, principalmente a Deus, pois sem ele, nada disso poderia existir.
\end{itemize}

Muito obrigado!

\begin{abstract}
O processamento distribuído, através das Grades computacionais, conta com uma grande infra-estrutura de redes. Esta infra-estrutura pode ser empregada em compartilhamento de programas, dados e serviços. Para gerenciar essa diversidade de recursos destacam-se os Sistemas Gerenciadores de Recursos (RMS), esses sistemas têm por função gerenciar de forma cooperativa e transparente os recursos geograficamente distribuídos considerando-os como pertencentes a um único computador. Neste contexto apresenta-se o protótipo AppMan baseado no modelo GRAND ({\bf G}rand {\bf R}obust {\bf A}pplicatio{\bf n} {\bf D}eployment). Esse protótipo não implementa por completo o que sugere o modelo. Dentre as características não implementadas, cita-se a integração com diferentes RMS. Através de estudos encontrou-se a Implementação DRMMA ({\bf D}istribuited {\bf R}esource {\bf M}anagement {\bf A}pplication {\bf A}PI) que visa integrar diferentes RMS. Este trabalho demonstra a viabilidade da integração do protótipo AppMan com o RMS PBS através da DRMAA. A versão da DRMAA usada foi o pacote na linguagem Java que é a mesma linguagem do protótipo. Com a integração, foi possível submeter aplicações com um número de tarefas do que na implementação anterior. Os testes comprovaram que, apesar da necessidade de inúmeras melhorias no protótipo, a integração é possível com pouco intrusão no código atual. A versão da DRMAA usada foi o pacote na linguagem Java que é a mesma linguagem do protótipo.
\end{abstract}

\begin{englishabstract}{Integration of the AppMan Application Management for the Grid Environment with the Resource Management Systems PBS}{Grid Computing, Resource Management}
Distributed Processing in Computational Grids can use a large  networks infrastructure. This infrastructure can be used in sharing of programs, data, and services. We can use a Resource Management System (RMS) to manage this diversity of resources.  Such a system can manage in a cooperative and transparent way the resources geographically distributed considering them as belonging to a single computer. In this context, there is a prototype model called AppMan based on the GRAND ({\bf G}rand {\bf R}obust {\bf A}pplicatio{\bf n} {\bf D}eployment) model. This prototype doesn't fully implement the GRAND model. The GRAND model, missing specifically the integration with different RMS. Through studies we found the DRMAA ({\bf D}istribuited {\bf R}esource {\bf M}anagement {\bf A}pplication {\bf A}PI) implementation which aims to integrate different RMS. This work demonstrates the feasibility of integrating the prototype AppMan with the PBS RMS through DRMAA. Tests have shown that, despite the need for numerous improvements in the prototype, integration is possible with little intrusion in the current code. The DRMAA version used was a Java language API that is the same language of the prototype.
\end{englishabstract}