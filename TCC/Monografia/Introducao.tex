\chapter{Introdução}
\label{cap:introducao}

Grades computacionais (grid computing) é uma das formas mais recentes de ambiente para processamento geograficamente distribuído, que conta com uma grande infra-estrutura de redes e pode ser empregada em troca de programas, dados e serviços. Segundo Dantas \cite{Dantas2005}, pode-se dizer, também, que representa uma forma estendida dos serviços Web permitindo que recursos computacionais possam ser compartilhados. Podemos definir grades como uma plataforma computacional heterogênea distribuída geograficamente fornecendo serviços e recursos às organizações participantes da plataforma \cite{Dantas2005}. 
       
Um sistema de gerenciamento de recursos (\emph{Resource Management System} - RMS) é a parte central de um sistema distribuído fornecendo um mecanismo de enfileiramento de tarefas, políticas de escalonamento, esquemas de prioridades e monitoramento de recursos proporcionando controles adicionais sobre inicialização, escalonamento e execução de tarefas. Também coordena a distribuição dessas tarefas entre as diferentes máquinas em uma rede \cite{condor2007, Bayucan2007, Mangan2006}. Devido a heterogeneidade das grades alguns problemas são apresentados, tais como, a alocação dos nós para um grande número de tarefas, gerenciamento de dados e sobrecarga em nós de submissão. O modelo de gerenciamento de aplicações denominado GRAND (\emph{Grid Robust Application Deployment}) \cite{Mangan2006} visa permitir um particionamento flexível e utilizar uma hierarquia de gerenciadores que realizam a submissão das tarefas. Baseado nesse modelo um protótipo, AppMan (\emph{Application Manager}) \cite{Vargas2003}, foi implementado objetivando garantir o escalonamento das tarefas bem como a autorização e autenticação para tarefas executadas. Ele foi avaliado apresentando bons resultados referente ao gerenciamento de dados e serviços. Esse protótipo consiste em um gerenciador de aplicações que dispara e controla cada aplicação nos nós baseando-se nas informações indicadas pelo gerenciador de submissão que tem como função, além da já citada, criar e monitorar os gerenciadores de tarefas. Os gerenciadores de tarefas são responsáveis pela comunicação com o escalonador de um determinado domínio garantindo a execução remota e ordem das tarefas de acordo com a dependência de dados \cite{Mangan2006}.

O modelo GRAND permitiria que qualquer sistema gerenciador de recurso fosse usado nos nós, porém o AppMan funciona unicamente com seu próprio sistema de gerenciamento de recursos. Nenhum estudo mais detalhado foi realizado até o momento de como possibilitar que o protótipo suporte a integração com diferentes RMSs.

Uma interface de aplicação (\emph{Application Program Interface} - API) denominada DRMAA (\emph{Distributed Resource Management Application} API) foi desenvolvida com o objetivo de facilitar a integração das aplicações para diferentes RMSs \cite{Rajic2004}.

Este trabalho pretende avaliar se a especificação DRMAA atende as necessidades do AppMan bem como realizar a integração do AppMan ao menos com um RMS.

Colocar um parágrafo que explicará o restanto da monografia.
