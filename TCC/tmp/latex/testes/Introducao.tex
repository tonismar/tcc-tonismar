\section*{Introdução}
\label{cap:introducao}

Atualmente o uso de redes de computadores tem aumentado exponencialmente. Muitas dessas redes são distribuídas de forma geograficamente separadas precisando de uma complexa infra-estrutura de software e hardware para gerenciá-las e conectá-las. Dentre as diversas soluções existe a grade computacional (\emph{grid computing}).
Segundo Dantas [aput 1], pose dizer, também, que representa uma forma estendida dos serviços web permitindo que recursos computacionais possam ser compartilhados.
Defini-se grades como uma plataforma computacional heterogênia distribuída geograficamente fornecendo serviços e recursos às organizações participantes da plataforma[apud 1].
[apud 1] O Global Grid Forum (GGF) uma comunidade fórum com milhares de indivíduos representando mais de 400 organizações em mais de 50 países criou e documentou especificações técnicas e experiências de usuários. O GGF definiu grades computacionais como um ambiente persistente o qual abilita aplicações para integrar intrumentos, disponibilizar informações em locações difusas. Desde lá não é a única e precisa definição para o conceito de grades. [Foster 3] Define um sistema em grade propondo um \emph{checklist} de três pontos.
