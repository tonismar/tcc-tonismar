%------------------------------------------------------------------------------------------------
% \imagemfloat{arquivo}{tamanho}{legenda}{label}
%------------------------------------------------------------------------------------------------
\newcommand{\imagemfloat}[4]{
    \begin{figure}[htb]
          \begin{center}\resizebox*{!}{#2cm}{\includegraphics{#1}\par}
          \end{center}
          \caption{#3}
          \label{#4}
    \end{figure}
}
\newcommand{\figura}[4]{
    \begin{figure}[H]
    \begin{center}\includegraphics[scale=#2]{imagens/#1}\end{center}
    \caption{\label{#3}#4}
    \end{figure}   
}
%------------------------------------------------------------------------------------------------
% \simb[entrada na lista de s�mbolos]{s�mbolo}:
%   Escreve o simbolo no texto e uma entrada na Lista de S�mbolos.
%   Se o par�metro opcional e omitido, usa-se o par�metro obrigat�rio.
%------------------------------------------------------------------------------------------------
\newcommand{\simb}[2][]{%
  \ifthenelse{\equal{#1}{}}
    {\addcontentsline{los}{simbolo}{\ensuremath{#2}}}
    {\addcontentsline{los}{simbolo}{#1}}
  \ensuremath{#2}
}

\makeatletter
%------------------------------------------------------------------------------------------------
% \listadesimbolos: comando que imprime a lista de simbolos
%------------------------------------------------------------------------------------------------
\newcommand{\listadesimbolos}{
  \pretextualchapter{Lista de S�mbolos}
  {\setlength{\parindent}{0cm}
   \@starttoc{los}}}

\newcommand\listabbrname{Lista de S\'imbolos e Abreviaturas}
% => obtido do padrao do II-UFRGS (utug-l)
\newenvironment{listofabbrv}[1]{
        \chapter*{\listabbrvname}
        \begin{list}{\textbf{??}}{
                \settowidth{\labelwidth}{#1}
                \setlength{\labelsep}{1em}
                \setlength{\itemindent}{0mm}
                \setlength{\leftmargin}{\labelwidth}
                \addtolength{\leftmargin}{\labelsep}
                \setlength{\rightmargin}{0mm}
                \setlength{\itemsep}{.1\baselineskip}
                \renewcommand{\makelabel}[1]{\makebox[\labelwidth][l]{##1}}
        }
}{
        \end{list}
}

%------------------------------------------------------------------------------------------------
% como a entrada ser� impressa
\newcommand\l@simbolo[2]{\par #1, p.\thinspace#2}
%------------------------------------------------------------------------------------------------
% Substituto para folha de rosto.
%\begin{titlepage}
%  \vfill
%  \begin{center}
%    {\large Ricardo Farias Bidart Piccoli} \\[5cm]
%    {\Huge Um algoritmo para quantiza��o de imagens coloridas baseado em preserva��o
%de contraste}\\[1cm]
%     \hspace{.45\textwidth}
%      \begin{minipage}{.5\textwidth}
%           \begin{espacosimples}
%                Trabalho de conclus�o apresentado in order to... duh
%           \end{espacosimples}
%      \end{minipage}
%  \vfill
%  Junho de 2006
%  \end{center}
%\end{titlepage}

\makeatother

\providecommand{\tabularnewline}{\\}
\floatstyle{ruled}
\newfloat{algorithm}{tbp}{loa}
\floatname{algorithm}{Algoritmo}

 \newenvironment{lyxcode}
   {\begin{list}{}{
     \setlength{\rightmargin}{\leftmargin}
     \setlength{\listparindent}{0pt}% needed for AMS classes
     \raggedright
     \setlength{\itemsep}{0pt}
     \setlength{\parsep}{0pt}
   %  \normalfont\ttfamily
}%
    \item[]}
   {\end{list}}

\renewcommand{\ABNTchapterfont}{\bfseries\sffamily\fontseries{sbc}\selectfont}
\renewcommand{\ABNTsectionfont}{\bfseries\sffamily}
% \fontfamily{cmss}\fontseries{sbc}\selectfont.
