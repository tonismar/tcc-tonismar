\chapter{Dados Experimentais}
\label{cap:Dados_Experimentais}

Este cap�tulo apresenta os resultados dos ...


Para a m�dia foi calculada a m�dia aritm�tica simples e para o desvio foi calculado atrav�s da f�rmula do desvio padr�o (n�o-polarizado ou n-1) escrita assim:

\begin{center}
\begin{math}
\sqrt{ \frac{n\sum x^{2} - (\sum x)^{2}}{n(n-1)}}
\end{math}
\end{center}



\section{\label{sec_Intrusao_Maquina}Intrus�o na M�quina}


A Tabela \ref{tab:intrusao_maquina} mostra o ...


\begin{table} [hbtp]
\begin{center}
\caption{Intrus�o nas M�quinas}
\label{tab:intrusao_maquina}
%\begin{footnotesize}
\begin{scriptsize}
\begin{tabular}{l|l|l|l|l|l}
	\hline
		{\bf Equipamento} & \multicolumn{3}{c|}{\bf Ocupa��o CPU} &  {\bf Mem�ria} &  {\bf Tam. Arq. Gerado}\\
				  &	M�x	&  M�d	& 	Desvio	  & 	  	   & 			    \\
	\hline
		Equip 1		  &  50 \%	& 24 \%	 & 	19 \%	  &   3 MB	   & 	17 Kbytes	    \\ \hline
		Equip 10	  &  49	\%	& 23 \%	 & 	16 \%	  &   5 MB	   & 	16 Kbytes	    \\ \hline \hline
		\bf{M�dia}	  &  49,7 \%	& 23,2 \%&	17,5 \%	  &  5,6 MB	   &	16,2 Kbytes	    \\ \hline
		\bf{Desvio Padr�o}&  0,48 \%	& 1,03 \%&	1,64 \%	  &  1,64 MB	   &	0,78 Kbytes	    \\ \hline



\end{tabular}
%\end{footnotesize}
\end{scriptsize}
\end{center}
\end{table}


