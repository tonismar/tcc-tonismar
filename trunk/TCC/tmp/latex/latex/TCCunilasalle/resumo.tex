%dtl%\begin{resumo}
%pkvm%
\begin{abstract}
Em um ambiente de computa��o em grade, ...  Assim, este trabalho busca um modelo de monitoramento de recursos de
hardware e software, que contemple as quest�es de escalabilidade e
heterogeneidade,para ambientes de grade. 
Um prot�tipo foi implementado ... . As avalia��es mostraram ....






%O Sistema Operacional atendido pelo prot�tipo � Windows. 
%Uma vez com o prot�tipo finalizado, foi utilizado como ambiente de teste o laborat�rio do Banrisul \abbrev{BANRISUL}{Banco %do %Estado do Rio Grande do Sul}.Os resultados da mo\-ni\-to\-ra\-��o ser�o disponibilizados no formato XML %\abbrev{XML}{\emph{eXtensible Markup Language}} para facilitar a integra��o com outras ferramentas.

%Posteriormente, deseja-se montar uma grade de teste envolvendo m�quinas do laborat�rio 24h do Unilasalle, da UFRJ %\abbrev{UFRJ}{Universidade Federal do Rio de Janeiro} e da UFLA\abbrev{UFLA}{Universidade Federal de Lavras} para teste do %sensor Linux. Os resultados da mo\-ni\-to\-ra\-��o ser�o 
%disponibilizados no formato XML \abbrev{XML}{\emph{eXtensible Markup Language}} para facilitar 
%a integra��o com outras ferramentas.
%dtl%\end{resumo}
%dtl%\begin{abstract}
\end{abstract}

% resumo na outra l�ngua
% como parametros devem ser passados o titulo e as palavras-chave
% na outra l�ngua, separadas por v�rgulas

%\selectlanguage{english}

\begin{englishabstract}{Resource Monitoring in Grid Environments}{Grid Computing, Monitoring, Web Services}
In a grid computing environment, ...
Thus, this work aims at proposing a hardware and software resource
monitoring model for grid environment, which deals with scalability
and heterogeneity issues.
A prototype was implemented ... The evaluation showed ...
\end{englishabstract}

%\selectlanguage{brazilian}

