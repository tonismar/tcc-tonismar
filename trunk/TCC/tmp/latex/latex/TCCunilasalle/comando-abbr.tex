%------------------------------------------------------------------------------------------------
% \simb[entrada na lista de s�mbolos]{s�mbolo}:
%   Escreve o simbolo no texto e uma entrada na Lista de S�mbolos.
%   Se o par�metro opcional e omitido, usa-se o par�metro obrigat�rio.
%------------------------------------------------------------------------------------------------
\newcommand{\simb}[2][]{%
  \ifthenelse{\equal{#1}{}}
    {\addcontentsline{los}{simbolo}{\ensuremath{#2}}}
    {\addcontentsline{los}{simbolo}{#1}}
  \ensuremath{#2}
}

\makeatletter
%------------------------------------------------------------------------------------------------
% \listadesimbolos: comando que imprime a lista de simbolos
%------------------------------------------------------------------------------------------------
\newcommand{\listadesimbolos}{
  \pretextualchapter{Lista de S�mbolos}
  {\setlength{\parindent}{0cm}
   \@starttoc{los}}}

\newcommand\listabbrname{Lista de S\'imbolos e Abreviaturas}
\newcommand\listofabbreviations{%
    \chapter*{\listabbrname
      \@mkboth{\MakeUppercase\listabbrname\space\draftdate}%
              {\MakeUppercase\listabbrname\space\draftdate}}%
     % \addcontentsline{toc}{chapter}{\MakeUppercase\listabbrname}%
    \@starttoc{lob}%
    }
\newcommand\abbrev[2]{%
                                        \def\({$}%
                                        \def\){$}%
      \addcontentsline{lob}{section}{%
                                                                \rm%
                        \protect\parbox[t]{.2\textwidth}{\bf #1}%
                   \hspace{0.025\textwidth}%
                   \protect\parbox[t]{.6\textwidth}{#2}%
        \vspace{2mm}\hspace{.1\textwidth}}%
      }

%------------------------------------------------------------------------------------------------
% como a entrada ser� impressa
\newcommand\l@simbolo[2]{\par #1, p.\thinspace#2}
