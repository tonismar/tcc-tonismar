% linhas come�ando com porcentagem s�o coment�rios
% O exemplo foi obtido em http://wiki.cecm.usp.br/wiki/LATEX 
% e levemente alterado e acrescido de coment�rios

% document class define o tipo de documento, que neste caso � um artigo
\documentclass [12pt]{article}

% usepackage serve para incluir pacotes, 
% os abaixo s�o fundamentais para permitir o uso de acentos
\usepackage[portuguese]{babel}
\usepackage[latin1]{inputenc}

% essa linha inclui um pacote que permite adicionar figuras
\usepackage[dvips]{graphicx}

% daqui para baixo come�a o documento!
\begin{document}

\author {Nome do Autor}

\title {T�tulo do Trabalho}
\maketitle

% note que para criar uma se��o basta colocar o comando section
\section{Primeira Se��o}

Primeiro paragrafo da se��o 1.

Para mudar de paragrafo voc� tem que pular uma linha.
Simplesmente dar enter n�o inicia novo par�grafo.

A seguir um exemplo de formata��o, a cria��o de itens:

\begin{itemize}

\item primeiro item
\item segundo item
\item terceiro item

\end{itemize}

  
\section{Segunda Se��o}

� s� escrever o texto que o latex se vira para formatar.

\end{document}