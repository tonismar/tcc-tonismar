
%
% inicio do documento
%

% capa
\maketoppage

% folha de rosto
%\maketitle

\thetitlepage
\makeappage

% dedicatoria
%\clearpage
%\begin{flushright}
%\mbox{}\vfill
%{\sffamily\itshape
%``If I have seen farther than others,\\
%it is because I stood on the shoulders of giants.''\\}
%--- \textsc{Sir~Isaac Newton}
%\end{flushright}

% agradecimentos
\chapter*{Agradecimentos}

\begin{abstract}
O processamento distribuído através das Grades computacionais, conta com uma grande infra-estrutura de redes. Esta infra-estrutura pode ser empregada em troca de programas, dados e serviços. Para gerenciar essa diversidade de aplicações destacam-se os Sistemas Gerenciadores de Recursos (RMS), esses sistemas tem por função gerenciar de forma cooperativa e transparente os recursos geograficamente distribuídos considerando-os como pertencentes de um único computador. Neste contexto apresenta-se o protótipo AppMan baseado no modelo GRAND ({\bf G}rand {\bf R}obust {\bf A}pplicatio{\bf n} {\bf D}eployment). Esse protótipo não implementa por completo o que sugere o modelo, dentre das características não implementadas, cita-se a integração com diferentes RMS. Através de estudos encontrou-se a Implementação DRMMA ({\bf D}istribuited {\bf R}esource {\bf M}anagement {\bf A}pplication {\bf A}PI) que visa integrar diferentes RMS. Este trabalho demonstra a viabilidade da integração do protótipo AppMan com o RMS PBS através da DRMAA. Os testes comprovaram que, apesar da necessidade de inúmeras melhorias no protótipo, a integração é possível com pouco intrusão no código atual. A versão da DRMAA usada foi o pacote na linguagem Java que é a mesma linguagem do protótipo.
\end{abstract}

\begin{englishabstract}{}{Grid Computing, Resource Management}
The processing distributed through Grid Computing, has a large infrastructure networks. This infrastructure can be used in exchange of programs, data and services. To manage this diversity of applications are the Resource Management System (RMS), these systems is to manage in a cooperative and transparent the resource geographically distributed considering them as belonging to a single computer. In this context is theconsidering them as belonging to a single computer. In this context is the prototype model based on AppMan GRAND ({\bf G}rand {\bf R}obust {\bf A}pplicatio{\bf n} {\bf D}eployment). This prototype not completely implemented, refers to the integration with different RMS. Through studies found himself the Implementation DRMAA ({\bf D}istribuited {\bf R}esource {\bf M}anagement {\bf A}pplication {\bf A}PI) which aims to integrate different RMS. This resource demonstrates the feasibility of integrating the prototype AppMan with the RMS PBS through DRMAA. Tests have shown that, despite the need for numerous improvements in the prototype, integration is possible with little intrusion in the current code. The DRMAA version of the package was used in the Java language that is the same language of the prototype.
\end{englishabstract}