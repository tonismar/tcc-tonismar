\chapter{Conclusão}
\label{cap:conclusao}

O presente trabalho permitiu a integração do AppMan com o RMS PBS. Deste modo o principal objetivo deste trabalho foi atingido. Foi possível a integração usando a especificação DRMAA implementada para o GridWay permitindo comprovar que essa implementação fornece subsídios suficientes para que o AppMan futuramente seja integrado a outros RMS. Um dos pontos positivos nessa integração deve-se ao fato de sua integração não necessitar uma intrusão significativa no código do RMS em questão.  

O projeto GRAND, atualmente, possui uma série de cooperações informais com várias instituições de pesquisa como, UFRGS, UFRJ, UFLA e Universidade do Porto. Além das contribuições já existentes no projeto ao qual esta pesquisa está inserida, a integração do AppMan com PBS através da DRMAA aumenta a perspectiva de novos domínios administrativos integrarem-se na grade ampliando a escalabilidade. Além disso, esta implementação possibilita uma maior quantidade/diversidade de testes para o modelo GRAND. 

A presente pesquisa também contribui com uma melhora e atualização na instalação do protótipo e o ambiente necessário para executá-lo. Além de comentários no código, uma série de documentos foram gerados:

\begin{itemize}
	\item Apêndice ~\ref{anexo:pbs-linux} Instalação do RMS PBS.
	\item Apêndice ~\ref{anexo:ldap} Instalação do servidor LDAP.
	\item Apêndice ~\ref{anexo:appman} Instalação e aquisição do protótipo AppMan.
	\item Apêndice ~\ref{anexo:nfs} Instalação e configuração do servidor NFS.
\end{itemize}

Como principal trabalho futuro destaca-se, a implementação de alguma alternativa para o NFS como um sistema de transferência de arquivo ou pNFS \cite{pNFS}. Esta carência no protótipo impossibilitou testes com outros domínios administrativos em particular com o LNCC. Outros trabalhos que trariam benefícios interessantes seria a integração com outros gerenciamento de recursos principalmente o Condor. Uma melhora no algoritmo de escalonamento também seria desejável, principalmente na detecção de termino de execução de uma tarefa.