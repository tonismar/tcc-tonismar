\chapter{Instalação PBS no Linux}
\label{anexo:pbs-linux}

\begin{enumerate}
	\item \textbf{Configurando RSH/SSH}
	
		Ssh precisa ser configurado para conexão sem a necessidade de senhas na comunicação entre servidor e os nós. Para isto foi usado o programa \emph{ssh-keygen} que gera duas chaves, uma simétrica e uma assimétrica. 
		\begin{scriptsize}
		\begin{verbatim}
			$ cd ~/.ssh/
			$ ssh-keygen -b 1024 -t rsa		
		\end{verbatim}
		\end{scriptsize}
		Após geração das chaves enviou-se a chave pública para os nós.
		\begin{scriptsize}
		\begin{verbatim}
			$ scp id_rsa.pub no1@ip_no1:/home/tonismar/.ssh/
		\end{verbatim}		
		\end{scriptsize}			
		O comando acima foi repetido para todos os nós alterando o nome do host e o ip.
		Após isto, em cada nó que recebeu a chave público executa-se o seguinte comando em cada um dos nós:
		\begin{scriptsize}
		\begin{verbatim}
			$ cat id_rsa.pub > authorized_keys
		\end{verbatim}
		\end{scriptsize}
		Para testar foi feito ssh para um dos nós e comprovado que não foi solicitado senha de autenticação.
	
	\item \textbf{Download OpenPBS}
		O PBS instalado foi a versão 2.3.16 adquirido no site www.openpbs.org.
		
	\item \textbf{Descompactação PBS}
		
		\begin{scriptsize}
		\begin{verbatim}
			# cd path/onde/esta/o/pacote/baixado/
			# tar -zxvf OpenPBS_2_3_16.tar.gz
			# cd OpenPBS_2_3_16/
		\end{verbatim}
		\end{scriptsize}	
		
	\item \textbf{Aplicação do \emph{patch}}
		
		Conforme documentação, foi necessário aplicar o \emph{patch} disponível. Ainda no diretório raiz do OpenPBS executar os seguintes comandos:
		\begin{scriptsize}
		\begin{verbatim}
			# cd /path/OpenPBS_2_3_16
			# patch -p1 -b < pbs.patch
		\end{verbatim}
		\end{scriptsize}
		
	\item \textbf{Compile PBS}
		\begin{itemize}
			\item Nó principal (servidor)
				\begin{scriptsize}
				\begin{verbatim}
					# mkdir /path/OpenPBS_2_3_16/head
					# cd /path/OpenPBS_2_3_16/head
					# ../configure --with-scp --disable-gui --set-server-home=/var/spool/PBS \
					--set-default-server=<nome do host do servidor>
					# make
					# make install
				\end{verbatim}
				\end{scriptsize}
				
			\item Demais nós
				\begin{scriptsize}
				\begin{verbatim}
					# mkdir /path/OpenPBS_2_3_16/compute
					# cd /path/OpenPBS_2_3_16/compute
					# ../configure --with-scp --disable-gui --set-server-home=/var/spool/PBS \
						--disable-server --set-default-server=<nome do host nó> --set-sched=no
					# make
					# rsh <nome do host nó>
				\end{verbatim}
				\end{scriptsize}
		\end{itemize}
		
	\item \textbf{Configuração}
		\begin{itemize}
			\item Arquivo de descrição dos nós. Contém a listas dos nós que devem estar disponíveis.
				\begin{scriptsize}
				\begin{verbatim}
					neo np=2 (np number processor)
					desktop np=1
					tonismar-laptop np=1
				\end{verbatim}
				\end{scriptsize}
			\item Arquivo \textbf{\emph{config}} de configuração nos nós. A configuração deste arquivo faz com que todas mensagens (exceto depuração) sejam colocadas no nó \emph{neo}.
				\begin{scriptsize}
				\begin{verbatim}
					$logevent 0x0ff
					$clienthost neo
					$restricted neo
				\end{verbatim}
				\end{scriptsize}
			\item Configuração do servidor. Os comandos abaixo criam uma fila de execução chamada \emph{\textbf{workq}} que está ativa e inicializada. Está é a fila padrão do servidor.
				\begin{scriptsize}
				\begin{verbatim}
					# /usr/local/sbin/pbs_server -t create
					# qmgr
					
					>c q workq
					>s q workq queue_type=execution
					>s q workq enabled=true
					>s q workq started=true
					>s s scheduling=true
					>s s query_other_jobs=true
					>s s node_pack=false
					>s s log_events=511
					>s s scheduler_iteration=500
					>s s resources_default.neednodes=3
					>s s resources_default.nodect=3
					>s s resources_default.nodes=3
					>quit
				\end{verbatim}
				\end{scriptsize}
			\item Inicialização do Escalonador. Apenas no servidor é necessário a inicialização do Escalonador, através do comando:
				\begin{scriptsize}
				\begin{verbatim}
					# /usr/local/sbin/pbs_sched
				\end{verbatim}
				\end{scriptsize}
			\item Inicialização do Monitor. Para iniciar o Monitor nos nós e também no servidor usa-se o comando:
				\begin{scriptsize}
				\begin{verbatim}
					# /usr/local/sbin/pbs_mom
				\end{verbatim}
				\end{scriptsize}
		\end{itemize}
		
		\item \textbf{Testando PBS}
		
			Após o processo de instalação e configuração o teste para verificar o funcionamento pode ser feito através do \emph{script} abaixo:
			\begin{scriptsize}
			\begin{verbatim}
				#!/bin/sh
				#testpbs
				echo This is a test
				echo Today is `date`
				echo This is `hostname`
				echo The current working directory is `pwd`
				ls -alF /home
				uptime
			\end{verbatim}
			\end{scriptsize}
			e para submissão deste \emph{script} usa-se o comando:
			\begin{scriptsize}
			\begin{verbatim}
				$qsub testpbs
			\end{verbatim}
			\end{scriptsize}
			Após a execução da tarefa a saída é armazenda no mesmo diretório onde foi submetida.
\end{enumerate}