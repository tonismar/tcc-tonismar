\chapter{Gerenciamento de Aplicações em Grade}
\label{cap:gerenciamento}

A idéia de computação em grade para processamento de aplicações em paralelo veio, por consequência, dos inúmeros avanços no desempenho de redes de computadores.

Atualmente o uso dessas redes tem aumentado exponencialmente. Muitas dessas redes são distribuídas de forma geograficamente separadas precisando de uma complexa infra-estrutura de software e hardware para gerenciá-las e conectá-las. Dentre as diversas soluções existentes a grade computacional (grid computing) possui características que viabiliza essa conexão.

O Open Grid Forum (OGF) uma comunidade fórum com milhares de indivíduos representando mais de 400 organizações em mais de 50 países criou e documentou \cite{M.2002} especificações técnicas e experiências de usuários. O OGF definiu grades computacionais como um ambiente persistente o qual habilita aplicações para integrar instrumentos, disponibilizar informações em locações difusas. Desde lá esta não é a única e precisa definição para o conceito de grades. Foster \cite{Kesselman2001} define um sistema em grade propondo um \emph{checklist} de três pontos.

\begin{enumerate}
	\item coordenar recursos os quais não são direcionados para um controle central.
	\item usar protocolos e interfaces padronizados, abertos para propósitos gerais.
	\item oferecer QoS (qualidade de serviço) não triviais tais como: autenticação, escalonamento de tarefas, disponibilidade.
\end{enumerate}

Uma definição formal do que um sistema em grade pode prover foi definido por Foster et al. em \cite{Foster2002}. Focando na sua semântica, mostrando que grades não são apenas uma modificação de um sistema distribuído convencional. Podem apresentar recursos heterogênios como sensores e detectores e não apenas nós computacionais. Abaixo uma lista de aspectos que evidenciam uma grade computacional \cite{Cirne2002}:

\begin{itemize}
	\item heterogeneidade
	\item alta dispersão geográfica
	\item compartilhamento (não pode ser dedicado a uma única aplicação)
	\item múltiplos domínios administrativos (recursos de várias instituições)
	\item controle distribuído 
\end{itemize}

A grade deve estar preparada para lidar com todo o dinamismo e variabilidade, procurando obter a melhor performance possível adaptando-se ao cenário no momento.