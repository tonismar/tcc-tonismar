\section{Gerenciamento de Recursos}

Devido à grande escala, ampla distribuição e existência de múltiplos domínios administrativos, a construção de um escalonador de recursos para grades é praticamente inviável, até porque, convencer os administradores dos recursos que compõem a grade abrirem mão do controle dos seus recursos não é uma tarefa nada fácil. Escalonadores têm como características receber solicitações de vários usuários, arbitrando, portanto, entre os usuários, o uso dos recursos controlados.  

Casavant \cite{Thomas1996} considera escalonar como um problema de gerenciamento de recursos. Basicamente um mecanismo ou uma política usada para, eficientemente e efetivamente, gerenciar o acesso e uso de um determinado recurso. Porém, de acordo com o OGF's \cite{M.2002}, escalonamento é o processo de ordenar tarefas sobre os recursos computacionais e ordenar a comunicação entre as tarefas, assim sendo, ambas aplicações e sistemas devem ser escalonadas. 

O gerenciamento de recursos de um sistema centralizado possui informação completa e atualizada do status dos recursos gerenciados. Este difere do sistema distribuído, o qual não tem conhecimento global de recursos dificultando assim, o gerenciamento. O ambiente em grade introduz cinco desafios para o problema de gerenciamento de recursos em ambientes distribuídos \cite{Karl1998}:

\begin{enumerate}
\item autonomia: os recursos são, tipicamente propriedades e operados por diferentes organizações em diferentes domínios administrativos.
\item heterogeneidade: diferentes lugares podem usar diferentes sistemas de gerenciamento de recursos (RMS).
\item extender as políticas: suporte no desenvolvimento de nova aplicação de mecanismos de gerência num domínio específico, sem necessitar de mudanças no código instalado nos domínios participantes.
\item co-alocação: algumas aplicações tem necessidades de recursos os quais só podem ser satisfeitos apenas usando recursos simultâneos com vários domínios.
\item controle online: RMSs precisam suportar negociações para adaptar necessidades de aplicações para recursos disponíveis.
\end{enumerate}

Sistemas computacionais, na sua grande parte, falham ao tratar dois problemas \cite{Mangan2006}: 

\begin{itemize}
	\item gerenciamento e controle de um grande número de tarefas; 
	\item o balanceamento da carga da máquina de submissão e do tráfego da rede.
\end{itemize}

Uma distribuição dinâmica de dados e tarefas em uma hierarquia de gerenciadores poderia ajudar o gerenciamento de aplicações. O modelo GRAND baseado na submissão e controle particionados e hierárquicos foi proposto em \cite{Mangan2006}. Uma melhor explanação deste modelo será dada no próximo capítulo.

Grande parte das pesquisas sobre escalonamento de tarefas em grades seguem uma organização hierárquica ou centralizada tais como: Globus \cite{Foster1998}, Condor \cite{condor2007}, ISAM \cite{isam} e PBS \cite{Bayucan1998}.

Globus \cite{Globus}, um dos projetos mais referenciados na literatura, tem como principal software o Globus Toolkit (GT). Como o nome indica, o GT não é uma solução completa e sim um conjunto de serviços que podem ser combinados para a construção de um \emph{middleware} de grade. O Globus \cite{Foster1998} tem seu modelo de escalonamento centralizado. Não fornece suporte nativo as políticas de escalonamento mas permite que gerenciadores externos adicionem esta capacidade. O Globus (GT versão 3) oferece serviços de informação através de uma rede hierárquica chamada \emph{Metacomputing Directory Services} (MDS) \cite{Santos}. O gerenciamento de cada recurso é feito por uma instância do {\it Globus Resource Allocation Manager} (GRAM) \cite{Andrade2002}. GRAM é o responsável por instanciar, monitorar e reportar o estado das tarefas alocadas para o recurso. A GT4 \cite{Leon2006} disponibiliza tais serviços em uma arquitetura baseada em {\it Web Services} a {\it Open Grid Services Architecture} (OGSA) junto com {\it Web Services Resource Framework} (WSRF). A GT4 é focada na qualidade, robustez, facilidade de uso e documentação.

Um dos gerenciadores que podem ser integrados com o Globus é o PBS \cite{Bayucan1998}. O PBS é um RMS que tem por propósito prover controle adicionais sobre a execução de tarefas em batch. O sistema permite um domínio definir e implementar políticas tais como os tipos de recursos e como esses recursos podem ser usados por diferentes tarefas.

Outro projeto bastante importante é o Condor \cite{condor2007}, cujos trabalhos originais são voltados para redes de computadores, mas atualmente também contemplam {\it clusters} e grades. O Condor trabalha com a descoberta de recursos ociosos dentro de uma rede alocando esses recursos para execução das tarefas. Condor possui uma arquitetura de escalonamento centralizado, ou seja, uma máquina especial é responsável pelo escalonamento. Todas as máquinas podem submeter tarefas a máquina central que se responsabiliza de encontrar recursos disponíveis para execução da tarefa. Tanto o Condor quanto o Globus perdem pontos no quesito tolerância a falhas e escalabilidade devido ao fato de terem um controle centralizado onde um problema na máquina central comprometeria o sistema por inteiro. Além disso, para o Globus, são necessárias negociações com os donos de recursos além da necessidade do mapeamento dos clientes para usuários locais.

Já o projeto ISAM \cite{isam} possui uma arquitetura organizada na forma de células autônomas cooperativas. Sua proposta é fornecer uma infra-estrutura tanto para a construção quanto execução de aplicações pervasivas \cite {isam}. Concebida para habilitar as aplicações a obter informações do ambiente onde executam e se adaptam às alterações que ocorrem durante o transcurso da execução. O ISAM, diferente do Globus e do Condor possui um modelo de escalonador de tarefas descentralizado ajudando o sistema alcançar um bom nível de tolerância a falhas e escalabilidade.

O próximo capítulo faz uma análise sobre o modelo GRAND, suas características bem como o protótipo AppMan desenvolvido nos padrões do modelo GRAND.