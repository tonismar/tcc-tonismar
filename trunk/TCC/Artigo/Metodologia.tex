\section{Metodologia}
\label{cap:metodologia}

Para verificar a viabilidade da exportação de tarefas do AppMan para RMS diferentes, um estudo da especificação DRMAA será realizado visando o desenvolvimento desta integração. Será realizado, também, um estudo aprofundado no modelo GRAND onde o protótipo foi desenvolvido e um minucioso estudo da implementação atual do AppMan.

O desenvolvimento proposto na integração será feito na linguagem Java que foi a linguagem de desenvolvimento utilizada no AppMan visando a portabilidade. Além disso algumas comparações através de métricas para avaliar as vantagens e desvantagens com o próprio escalonador do AppMan serão analisadas.

O método de pesquisa utilizado está entre um experimento, baseado no fato de que o modelo GRAND espera a integração com qualquer RMS e também em um estudo de caso em função da análise da implementação proposta.

Até o presente momento está sendo feito um estudo sobre a DRMAA e onde ela já foi empregada verificando suas fases bem como o resultado dessas implementações \cite{Templeton}, \cite{Llorente2005} e \cite{Haas2004}.

Estudos detalhados no diagrama de classes do AppMan auxiliarão na escolha da melhor forma de implementação da integração com a DRMAA.

Os testes e avaliações serão baseados nos mesmos ambientes onde foram feitos os experimentos do AppMan \cite{Mangan2006}. 

Além do estudo da DRMAA, serão avaliadas outras formas de integração entre RMS.
