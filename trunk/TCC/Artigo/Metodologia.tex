\section{Metodologia}
\label{cap:metodologia}

Para verificar a viabilidade da exportação de tarefas do AppMan para RMS diferentes será realizado um estudo aprofundado no modelo GRAND onde o protótipo foi desenvolvido.

O método de pesquisa utilizado está entre um experimento, baseado no fato de que o modelo GRAND espera a integração com qualquer RMS e também em um estudo de caso em função da análise da implementação proposta.

O desenvolvimento proposto na integração será feito na linguagem Java que foi a linguagem de desenvolvimento utilizada no AppMan visando a portabilidade. Além disso, algumas comparações através de métricas para avaliar as vantagens e desvantagens com o próprio escalonador do AppMan serão analisadas.

Até o presente momento está sendo feito um estudo da DRMAA e onde ela já foi empregada verificando suas fases bem como o resultado dessas implementações \cite{Templeton}, \cite{Llorente2005} e \cite{Haas2004}. Também foi executado todos procedimentos de instalação do ambiente EXEHDA/ISAM e do AppMan. Alguns problemas foram encontrados com o servidor \emph{ Lightweight Directory Access Protocol} (LDAP) necessário para o funcionamento do EXEHDA e estão sendo sanados. Uma engenharia reversa das classes do projeto AppMan gerando o diagrama UML foi feita para facilitar o estudo da integração com a DRMAA.

Os testes e avaliações serão baseados nos mesmos ambientes onde foram feitos os experimentos do AppMan \cite{Mangan2006}. 

Além do estudo da DRMAA, é pretendido avaliar outras formas de integração entre RMS.
